\chapter{Introduction}

\section{Motivation}

Anomaly detection in medical images is an active area of research with many potential benefits for healthcare. Identifying anomalies in images such as MRI scans can assist clinicians in detecting conditions earlier and initiating treatment more rapidly. However, anomaly detection faces considerable challenges, especially in domains where labeled data is scarce, and normal anatomy exhibits high inter-subject variability.

[mention unsupervised]Traditional machine learning methods often fail to identify anomalies amidst diverse normal variations.

Recent methods based on denoising score matching provide a promising solution. By estimating the score (gradient of the log-density), score matching enables precise quantification of how likely samples are under the data distribution without requiring anomalous samples for training. Conceptually, a score is a vector field that points in the direction where the likelihood increases the most. In the context of anomaly detection in medical imaging, score matching offers a powerful, yet under-explored, tool to discern pathological patterns from normal variations.

I posit that a multiscale analysis of score estimates can effectively identify anomalies stemming from multiple underlying factors. In this research I interpret "mutliscale" to be multiple noise levels of perturbation.
Intuitively, higher noise levels obscure local information forcing the model to learn more global patterns. 

By utilizing multiple noise levels during training, both global and local contextual features can be captured.

This multiscale capacity is particularly beneficial for anomaly detection in medical images, where anomalies may manifest in localized textures or global shape.


\section{Thesis Statement}

